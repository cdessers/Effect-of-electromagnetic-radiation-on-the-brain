\documentclass[12 pt]{report}

\usepackage[utf8]{inputenc}
\usepackage[T1]{fontenc}

\usepackage{amssymb,amsmath}
\usepackage{textcomp}
\usepackage{mathtools}
\usepackage{enumitem}
\usepackage{bm}	% In order to have bold italic in math mode
\usepackage{todonotes}	% In order to have clean todo's
\usepackage[labelfont=bf]{caption}
\usepackage{subcaption}
\usepackage{graphicx}
\usepackage{epstopdf}
\usepackage[colorlinks=true,linkcolor=red]{hyperref}
\usepackage{cleveref}


\begin{document}

\begin{titlepage}
    \vspace{-1cm}
    
    \begin{minipage}{0.6\textwidth}
        \begin{flushleft}
            \centering
            \textsc{University of Liège} \\
            \textsc{Faculty of Applied Sciences} \\
            \textsc{Academic Year 2017 - 2018}
        \end{flushleft}
    \end{minipage}

    \vspace{-2.6cm}
    \hspace{0.5\textwidth}
    
    \begin{minipage}[t]{\textwidth}
        \begin{flushright}
            \includegraphics[scale = 0.6]{uLIEGE_Faculte_SciencesAppliquees_Logo_CMJN_pos}
        \end{flushright}
    \end{minipage}

    \vspace{1cm}


    \begin{minipage}{\textwidth}
        \hspace{-0.7cm}\noindent\makebox[\linewidth]{\rule{\textwidth}{2pt}} 
    \end{minipage}

    \vspace{1cm}

    \begin{minipage}{\textwidth}
        \begin{center}
           \hspace{-0.8cm}\Huge{\textsc{\textbf{MATH0471-Multiphysics Integrated Computational Project}}}\\
            \vspace{0.3cm}
            \hspace{-0.8cm}\textsc{C. GEUZAINE \\ R. Boman} \\
        \end{center}
    \end{minipage}

    \vspace{1cm}

    \begin{minipage}{\textwidth}
        \hspace{-0.7cm}\noindent\makebox[\linewidth]{\rule{\textwidth}{0.2pt}}
    \end{minipage}

    \vspace{1cm}

    \begin{minipage}{\textwidth}
        \begin{center}
            \hspace{-1cm}\Huge{\textbf{Intermediate report 1}} \\ 
            \vspace{0.5cm}
        \end{center}
    \end{minipage}

    \vspace{1cm}

    \begin{minipage}{\textwidth}
        \hspace{-0.7cm}\noindent\makebox[\linewidth]{\rule{\textwidth}{0.2pt}}
    \end{minipage}

    \vspace{1cm}

    \begin{minipage}{\textwidth}
        \begin{center}
            \hspace{-1cm}\large{\textsc{DELCOURT Arnaud}} \\
            \hspace{-1cm}\large{\textsc{DESSERS Christophe}} \\
            \hspace{-1cm}\large{\textsc{HEUCHAMPS Alexandre}} \\
            \hspace{-1cm}\large{\textsc{TOMASETTI Romin}} \\
            \vspace{1cm}
            \hspace{-1cm}\Large{\textsc{\today}} \\
        \end{center}
    \end{minipage}
    
    \vspace{1cm}
    
    \begin{minipage}{\textwidth}
        \hspace{-0.7cm}\noindent\makebox[\linewidth]{\rule{\textwidth}{2pt}}
    \end{minipage}
\end{titlepage}

\tableofcontents
\newpage
\listoffigures
\newpage
\listoftables
\newpage

\section{Questions to ask on February, 20}

\begin{itemize}
	%
	\item Do we have to take into account the frequency-dependency of the electromagnetic properties of the materials? (see \url{https://em.geosci.xyz/content/physical_properties/magnetic_permeability/magnetic_permeability_frequency_dependent.html})
	%
	\item Regarding the already-written code:
		\begin{itemize}
			\item Comments on the created classes?
			\item Is speed affected? Or we don't care?
			\item Especially, ask if Array\_3D\_Template<Node> nodes is a good way to start !
		\end{itemize}
	%
	\item Regarding testing, what are the simplest tests? Could we compare our results with other simulations? Do we have to compare with something that can be hand-solved?
	%
	\item Check our reasoning on the nodes. They all have:
		\begin{itemize}
			\item 3 coordinates (3 doubles) - How are we going to mesh ??? (we think we are obliged to keep rectangular parallelepiped because of the numerical scheme)
			%
			\item 6 fields (3 for $\vec{E}$ and 3 for $\vec{H}$ so 6 doubles)
			%
			\item 1 temperature (1 double)
			%
			\item 1 unsigned char for the ID of the material
			%
			\item $\Rightarrow$ That would make 10 doubles and one unsigned char or 10*4+1 bytes per node.
		\end{itemize}
\end{itemize}

\section{Database for human tissues}

\url{https://www.itis.ethz.ch/virtual-population/tissue-properties/downloads/database-v3-1/}

\section{Introduction}
In this first report, some of the test cases we used to test the code will be explained and the corresponding results will be analyzed. \\
The first test case that was developed is that of a thin wire on which only one component of the electric field is not set to zero. In theory, this configuration should produce a rotating magnetic field. \\
Another test case that was developed is that of an electric dipole antenna in which an electric field along one direction is set to a non zero value. \\
An important point should be pointed out here : no boundary conditions were imposed, which means that the presented results are meaningless when the different signals arrive to boundaries.

\section{Explanation of the test cases}
\subsection{Thin wire}
As was previously said, the study case here is that of a thin wire anly excited by only an electric field along one direction. As can be seen in \Cref{fig:THINWIRE}, the magnetic field generated in such a configuration is consistent with the physical intuition.
%
\begin{figure}
	\centering
	\includegraphics[scale=0.5]{}
	\caption{Results for the thin wire excited by a one-component electric field.}
	\label{fig:THINWIRE}
\end{figure}
%


\subsection{Dipole antenna}
Here, a dipole antenna is studied, but its air gap has a null dimension, meaning that the antenna is just a source. On this source, only a component of the electric field is set to a certain value whereas all the others are iposed to zero. Concerning the magnetic fields, their values are updated by using the classical 3 dimensional YEE algorithm. The results are visible in \Cref{fig:DIPOLEANTENNA}.
%
\begin{figure}
	\centering
	\includegraphics[scale=0.5]{}
	\caption{Results for the dipole antenna with a zero air gap.}
	\label{fig:DIPOLEANTENNA}
\end{figure}
%



\end{document}
